\documentclass[a4paper,ngerman,12pt,addpoints]{exam}
\usepackage{babel}
\usepackage[top=3.5cm,headsep=0.5cm,headheight=3cm,left=1.5cm,right=1.5cm]{geometry}
\usepackage[utf8]{inputenc}
\usepackage[T1]{fontenc}
\usepackage{lmodern}
\usepackage[ngerman]{babel}
\usepackage{booktabs} % schöne Tabellen
\usepackage{graphicx}
\usepackage{csquotes} % Anführungszeichen
\usepackage{paralist} % kompakte Aufzählungen
\usepackage{amsmath,textcomp,tikz,amssymb,amstext,mathtools,amsthm} %diverses
\usepackage{eso-pic} % Bilder im Hintergrund
\usepackage{mdframed}%Boxen


\pointpoints{Punkt}{Punkte}
\bonuspointpoints{Bonuspunkt}{Bonuspunkte}
\renewcommand{\solutiontitle}{\noindent\textbf{Lösung:}\enspace}
\chqword{Frage}
\chpgword{Seite}
\chpword{Punkte}
\chbpword{Bonus Punkte}
\chsword{Erreicht}
\chtword{Gesamt}
\hpword{Punkte:} % Punktetabelle
\hsword{Ergebnis:}
\hqword{Aufgabe:}
\htword{Summe:}

\newcommand{\setN}{\mathbb{N}}
\newcommand{\setZ}{\mathbb{Z}}
\newcommand{\setQ}{\mathbb{Q}}
\newcommand{\setR}{\mathbb{R}}
\newcommand{\setC}{\mathbb{C}}

\DeclareMathOperator{\Gal}{Gal}

\newcommand{\dozent}{Prof. Kebekus}
\newcommand{\fach}{Algebra und Zahlentheorie}

\pagestyle{headandfoot}
\firstpageheadrule
\firstpageheader{\dozent}{\fach}{WS 2018/19}
\firstpagefooter{}{}{\thepage\,/\,\numpages}




\begin{document}
		\begin{questions}
		\question[12]
			Sind folgende Polynome irreduzibel in $\setQ[x]$? Geben Sie alle Kriterien, die Sie verwenden, detailliert wieder.
			\begin{enumerate}[a)]
				\item $x^3+3x^2-5x-2$
				\item $x^{2019}+2018x^2+10x+14$
			\end{enumerate}
		\question[12]
			\begin{enumerate}[a)]
				\item Sei $K$ ein K"orper, $f \in K[x]$ irreduzibles Polynom.\\ $\mathrm{Z\kern-.3em\raise-0.5ex\hbox{Z}}$: $K[x]/(f)$ ist ein K"orper.
				\item $R, S$ kommutative Ringe mit 1, $I \subset S$ ein Ideal von $S$, $f: R \longrightarrow S$ Morphismus.\\
				$\mathrm{Z\kern-.3em\raise-0.5ex\hbox{Z}}$: $f^{-1}(I)$ ist Ideal von $R$ und $f^{-1}(I)$ ist Primideal $\Leftrightarrow I$ Primideal.
			\end{enumerate}
		\question[12]
			$K \subset L \subset M$ K"orpererweiterungen.
			\begin{enumerate}[a)]
				\item Definiere $[L:K]$.
				\item Gegeben $S \subset L$, definiere $K(S)$.
				\item Definiere: Was bedeutet $L/K$ algebraisch?
				\item Seien $L/K$ und $M/L$ algebraisch.\\ $\mathrm{Z\kern-.3em\raise-0.5ex\hbox{Z}}$: $M/K$ algebraisch.
			\end{enumerate}
		\question[12]
			Betrachte $\alpha = \sqrt{3}+\sqrt{7}$
			\begin{enumerate}[a)]
				\item Identifiziere die Galois-Gruppe $\Gal{\setQ(\alpha)/\setQ}$.
				\item Bestimme alle Zwischenk"orper $\setQ \subsetneq K \subsetneq \setQ(\alpha)$ und bestimme f"ur jeden Zwischenk"orper $K$ ein Element $\beta_K$, sodass $K=\setQ(\beta_K)$
			\end{enumerate}
		\question[12]
			\begin{enumerate}[a)]
				\item Definition: normale Untergruppe
				\item Beispiel (mit Beweis): Gruppe $G$ und normale Untergruppe $N \subset G$, wobei $N \neq \{1\}$ und $N \neq G$.
				\item $G$ eine endliche Gruppe, die auf einer Menge $M$ wirkt. W"ahle $x \in M$. \\
				$\mathrm{Z\kern-.3em\raise-0.5ex\hbox{Z}}$: $\#(G \cdot x) \mid \#G$
				\item $G$ eine Gruppe, $Z$ das Zentrum von $G$.\\
				$\mathrm{Z\kern-.3em\raise-0.5ex\hbox{Z}}$: $G/Z$ ist zyklisch $\Rightarrow G$ abelsch.
			\end{enumerate}
	\end{questions}
	\begin{center}
		\gradetable[h][questions]
	\end{center}
\end{document}